% Licencia: CC BY-SA 3.0

\PassOptionsToPackage{unicode}{hyperref}
\PassOptionsToPackage{naturalnames}{hyperref}
\documentclass[compress]{beamer}
\usepackage[spanish]{babel}
\usepackage[utf8]{inputenc}
\usepackage{lmodern}
\usepackage[T1]{fontenc}
\uselanguage{Spanish}
\languagepath{Spanish}

% Colores
\definecolor{backg}{HTML}{F2F2F2}
\definecolor{title}{HTML}{bdc3d1}
\definecolor{comments}{HTML}{BDBDBD}
\definecolor{keywords}{HTML}{08388c}
\definecolor{strings}{HTML}{FA5858}
\definecolor{links}{HTML}{2C2C95}

% Gráficos
\usepackage{tikz}
\usetikzlibrary{svg.path}

%% Temas %%
% Tema y tema de color
\usetheme{Dresden}
\usecolortheme{dolphin}
\useinnertheme{circles}
\setbeamercovered{invisible}
% Colores bloques
\setbeamercolor{block title}{bg=title,fg=links}
\setbeamercolor{block body}{bg=backg,fg=black}
\setbeamercolor{block title alerted}{fg=red!70!black,bg=title!92!red}
\setbeamercolor{block body alerted}{fg=black,bg=backg}
\setbeamercolor{block title example}{fg=green!70!black,bg=title!92!green}
\setbeamercolor{block body example}{fg=black,bg=backg}
\hypersetup{colorlinks,linkcolor=,urlcolor=links}
\setbeamertemplate{navigation symbols}{}
\setbeamertemplate{footline}{}

%% Título y otros %%
\title{Mensajería instantánea}
\subtitle{Fundamentos de Redes}
\author{Pablo Baeyens \and José Manuel Muñoz}
\date{}

\usepackage{listings}
\lstset{
  basicstyle=\footnotesize\tt,
  breakatwhitespace=false,
  breaklines=true,
  captionpos=b,
  extendedchars=true,
  language=Java,
  keywordstyle=\bf,
  showspaces=false,
  showstringspaces=false,
  showtabs=false,
  tabsize=2,
  extendedchars=true,
literate={á}{{\'a}}1 {é}{{\'e}}1 {í}{{\'i}}1 {ó}{{\'o}}1
         {ú}{{\'u}}1 {ñ}{{\~n}}1 {¡}{{\textexclamdown}}1
}

%% Presentación %%
\begin{document}

\begin{frame}
\titlepage
\end{frame}

\begin{frame}[fragile]{Mensajes}
  Los clientes se envían mensajes a través del servidor utilizando objetos:

\begin{lstlisting}
public class Mensaje implements java.io.Serializable {
  private int codigo; // Código
  private Date date; // Fecha de creación
  private String usuario; // Usuario que envía
  private String grupo; // Grupo al que se envía
  private String ruta; // Ruta de contenido
  private byte[] contenido; // Contenido del mensaje
  public boolean esDeChat; // Si es mensaje de chat
  //...
}
\end{lstlisting}
\end{frame}

\begin{frame}[fragile]{Hebra que envía}
  Hay dos hebras: una escucha los mensajes del servidor y otra envía.
\begin{lstlisting}
do {
mssg = scanner.nextLine().trim();
if (mssg != null && !mssg.equals("")) {
  if (esComando(mssg))
    persiste = parseComando(mssg);
  else {
    aEnviar = new Mensaje(Contactos.getActual(), mssg);
    outStream.writeObject(aEnviar);
  }
}
} while (persiste);
\end{lstlisting}
\end{frame}

\begin{frame}[fragile]{Hebra que escucha}
  La hebra que escucha lee mensajes y dependiendo del código los añade al historial de la conversación adecuada:
\begin{lstlisting}
while((m = (Mensaje) in.readObject()) != null) {
  parse(m);
\end{lstlisting}

Y en \texttt{parse}:

\begin{lstlisting}
switch(m.getCodigo()){
  //...
  case 1001:
  case 1002:
    esMensaje = true;
    Contactos.addMensaje(conv,m);
    break;
  case 1004:
    esMensaje = true;
    leerFichero(m);
    break;
  //...
}
\end{lstlisting}
\end{frame}

\begin{frame}[fragile]{Conversaciones}

  Las conversaciones pueden ser con grupos:
\begin{lstlisting}
public class Grupo {
private String groupName;
private ArrayList<Cliente> clientes;
private static final int maxClientes = 512;

//...

public void sendMessage(Mensaje mensaje) {
for (Cliente c:clientes)
  c.sendMessage(mensaje);
}
\end{lstlisting}
\end{frame}

\begin{frame}[fragile]{Envío de ficheros}
  Los mensajes pueden enviar ficheros:
  \begin{lstlisting}
private void leerFichero(Mensaje m) {
 FileOutputStream fos = new FileOutputStream("./Recibidos/" + m.getRuta());
 BufferedOutputStream bos = new BufferedOutputStream(fos);
 byte[] contenido = m.getRawContenido();
 bos.write(contenido, 0, contenido.length);
 bos.flush();
}
  \end{lstlisting}

\end{frame}

\end{document}
