% Licencia: CC BY-SA 3.0
%% Paquetes y configuración %%

% Beamer
\PassOptionsToPackage{unicode}{hyperref}
\PassOptionsToPackage{naturalnames}{hyperref}
\documentclass[compress]{beamer}

% Idioma
\usepackage[spanish]{babel}
\usepackage[utf8]{inputenc}
\usepackage{lmodern}
\usepackage[T1]{fontenc}
\uselanguage{Spanish}
\languagepath{Spanish}

% Matemáticas
\usepackage{amsfonts}
\usepackage{amsmath}
\usepackage{amssymb}
\usepackage{mathrsfs}
\usepackage{xfrac}
\newtheorem*{proposicion}{Proposición}
\theoremstyle{definition}
\newtheorem*{definicion}{Definición}
\newtheorem*{problema}{Problema}

% Colores
\definecolor{backg}{HTML}{F2F2F2}
\definecolor{title}{HTML}{bdc3d1}
\definecolor{comments}{HTML}{BDBDBD}
\definecolor{keywords}{HTML}{08388c}
\definecolor{strings}{HTML}{FA5858}
\definecolor{links}{HTML}{2C2C95}

% Gráficos
\usepackage{tikz}
\usetikzlibrary{svg.path}

%% Temas %%
% Tema y tema de color
\usetheme{Dresden}
\usecolortheme{dolphin}
\useinnertheme{circles}
\setbeamercovered{invisible}
% Colores bloques
\setbeamercolor{block title}{bg=title,fg=links}
\setbeamercolor{block body}{bg=backg,fg=black}
\setbeamercolor{block title alerted}{fg=red!70!black,bg=title!92!red}
\setbeamercolor{block body alerted}{fg=black,bg=backg}
\setbeamercolor{block title example}{fg=green!70!black,bg=title!92!green}
\setbeamercolor{block body example}{fg=black,bg=backg}
\hypersetup{colorlinks,linkcolor=,urlcolor=links}
\setbeamertemplate{navigation symbols}{}
\setbeamertemplate{footline}{}

% Evita warnings
\hfuzz=20pt
\vfuzz=20pt
\renewcommand\textbullet{\ensuremath{\bullet}}

%% Título y otros %%
\title{Mensajería instantánea}
\subtitle{Fundamentos de Redes}
\author{Pablo Baeyens \and José Manuel Muñoz}
\date{}

%% Presentación %%
\begin{document}

\begin{frame}
\titlepage
\end{frame}

\begin{frame}{Grupos}
  \begin{center}
  \begin{tikzpicture}[scale=0.2]
  \tikzstyle{every node}+=[inner sep=0pt]
  \draw [black] (17.9,-26.9) circle (3);
  \draw (17.9,-26.9) node {$C_1$};
  \draw [thick,fill=white] (28.6,-20.5) +(-3,-3) rectangle +(3,3) ;
  \draw (28.6,-20.5) node {$S$};
  \draw [black] (30.1,-8.9) circle (3);
  \draw (30.1,-8.9) node {$C_2$};
  \draw [black] (38.5,-13.8) circle (3);
  \draw (38.5,-13.8) node {$C_3$};
  \draw [black] (40.2,-21.7) circle (3);
  \draw (40.2,-21.7) node {$C_4$};
  \draw [black] (31.58,-20.81) -- (37.22,-21.39);
  \fill [black] (37.22,-21.39) -- (36.47,-20.81) -- (36.37,-21.81);
  \draw (34.2,-21.68) node [below] {$2$};
  \draw [black] (36.342,-15.881) arc (-49.31176:-62.5105:26.304);
  \fill [black] (36.34,-15.88) -- (35.41,-16.02) -- (36.06,-16.78);
  \draw (34.94,-18.22) node [below] {$2$};
  \draw [black] (28.98,-17.52) -- (29.72,-11.88);
  \fill [black] (29.72,-11.88) -- (29.12,-12.6) -- (30.11,-12.73);
  \draw (30.03,-14.84) node [right] {$2$};
  \draw [black] (19.695,-24.505) arc (136.48615:105.28374:12.881);
  \fill [black] (25.64,-20.95) -- (24.74,-20.68) -- (25,-21.64);
  \draw (21.42,-21.82) node [above] {$1$};
  \draw [black] (27.01,-23.033) arc (-40.00557:-78.22454:10.903);
  \fill [black] (20.88,-26.7) -- (21.77,-27.02) -- (21.56,-26.04);
  \draw (25.25,-25.88) node [below] {$2$};
  \end{tikzpicture}
  \end{center}
\end{frame}

\end{document}
